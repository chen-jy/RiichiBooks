%~~~~~~~~~~~~~~~~~~~~~~~~~~~~~~~~~~~~~~~~~~~~~~~~~
% Riichi Book 1, Preface
%~~~~~~~~~~~~~~~~~~~~~~~~~~~~~~~~~~~~~~~~~~~~~~~~~
\chapter{Preface}
\thispagestyle{empty}

When I moved to England in 2013, I was pleasantly surprised to learn that riichi mahjong (modern Japanese mahjong) is quite popular in Europe. In the past two years, I have had the pleasure of playing riichi in London, Guildford, Kent, Oxford, Aachen, Copenhagen,\\ Prague, and Vienna, along with players from Austria, China, Czech Republic, Denmark, Estonia, Finland, France, Germany, Italy, Japan, the Netherlands, Poland, Russia, Slovakia, Sweden, the UK, and the United States. 

\bigskip
European players have been remarkably successful in organizing tournaments open to anyone who plays the game. These tournaments --- held at least once a month somewhere in Europe --- are run by local mahjong players in each country under the auspices of the European Mahjong Association (EMA).\footnote{\url{http://mahjong-europe.org/}} \index{european@EMA}
Founded in 2005, EMA has been doing a fantastic job in maintaining common rule sets,%
\footnote{EMA's official rule book for riichi mahjong is available online at \url{http://mahjong-europe.org/portal/images/docs/Riichi-rules-2016-EN.pdf} (last revised in 2016). At the time of writing this book, EMA is in the process of revising the rule book. 
Explanations of EMA rules in this book are based on the revised rules. New rules will come into effect from April, 2016.
}
keeping a player ranking system, and doing many other useful things to promote the playing of mahjong across Europe. 

\bigskip
Although I have come across a few good players in Europe, I came to realize that a lot of players here are not very well-versed in the basic principles of competitive mahjong strategies. Of course, playing competitively is not the only way to enjoy the game.
I am also not claiming that I know the magic formula to win because there is no such thing. Nevertheless, there is a set of basic principles worth learning for any aspiring players. I believe the level of sophistication among European players could be much improved if these principles are more widely shared. Unfortunately, however, learning resources currently available for non-Japanese audience are somewhat limited.\footnote{There are already a few English books for beginners. There are also several excellent blog posts on technical details about mahjong strategies. However, there appears to be a huge gap between these two sets of resources. Introductory books do not cover strategies extensively, whereas blog posts tend to be too advanced even for intermediate players.}

\bigskip
I have thus decided to write a book on riichi mahjong strategies for European players, primarily with beginners and intermediate players in mind. I then ended up splitting the book into two volumes; Book I is intended for beginners and intermediate players ({\jap Tenhou} rank of 四段 or below), while Book II is meant for more advanced players. The two books are \emph{not} intended for complete novices who do not know how to play riichi mahjong.\footnote{If you want to learn how to play riichi, I'd recommend Barr (2009).} The target reader is anyone who has played riichi mahjong before and wants to improve their skills further. 
\index{Barr@Barr, Jenn}

\bigskip
I have three main goals in preparing these books. First, I will introduce a set of English terminology of riichi mahjong. 
``In beginning was Word,'' scripture tells us. Knowing the names of particular tile combinations, situations, and strategies will allow us to be conscious of them and to be able to talk about them with our fellow players after the game. 

\bigskip
My second goal is to introduce the principles of tile efficiency. 
Book I and Book II both cover tile efficiency, but at different levels. Book I offers an introduction to tile efficiency, covering very basic mechanisms only. I plan to cover more advanced materials in Book II. 
My third goal is to introduce a set of simple strategies regarding critical judgements such as whether or not to call {\jap riichi}, whether to push or to fold, and whether or not to meld.

\bigskip
A lot of the materials covered in the books were introduced to me through the writings of a notable Japanese mahjong player and manga author, Masa\-yuki Kata\-yama. Mr.~Kata\-yama is an accomplished riichi player and arguably the best mahjong manga author in the world. Some of the strategies introduced here are unabashedly stolen from Mr. Kata\-yama's masterpiece manga storybook $Uta\-hime$ $Obaka\-miiko$ (『打姫\-オバカ\-ミーコ』). 
I strongly encourage you to read it yourself if you read Japanese, although I realize that you would not be reading my book if you understood Japanese.
\index{Katayama@Katayama, Masayuki}

\bigskip
Another Japanese author whose work has been influential in the writing of Book I is Makoto Fukuchi. Mr.~Fukuchi is also a distinguished riichi player and the best-selling author of mahjong strategy books. A part of the exposition of the five-block method in Chapter \ref{ch:five}~is based on Mr.~Fukuchi's skillful explanation in his books. \index{Fukuchi@Fukuchi, Makoto}

\bigskip
I am also indebted to a lot of friends I have become acquainted with through mahjong in Europe. Philipp Martin has read an early draft of the book and provided me with valuable comments and encouragement. I am also grateful to Gemma Sakamoto, who has been hosting a monthly mahjong get-together in London. 
Finally, my thanks go to Ian Fraser, one of the founders of the UK Mahjong Association. 
Without the efforts of Ian and his team, I would not have been able to get to know so many fellow players in the UK and in Europe.

\bigskip
The cover photo (\copyright~Katar\'{i}na M\'{o}zov\'{a}) is from the 2015 Bratislava Riichi Open Tournament. I thank Katar\'{i}na and Riichi Mahjong Slovakia (especially Matej Laba\v{s}) for giving me their permission to use it.

\bigskip
After I made the book publicly available in January 2016, a lot of people have given me feedback on various aspects of the book. Based on their feedback, I corrected some terminology inconsistencies and typos. In particular, I thank David Clarke, Aaron Ebejer, Nicolas Giaconia, Grant Mahoney, Ting, Chris Rowe, Mike Liang, and Max Lu for their valuable inputs.

\vfill

\hfill Daina Chiba\\
\hfill London, UK\\
\hfill January 10, 2016\\
\hfill (updated on \today)

\section*{Plan of the book}

To improve your mahjong skills, you need not only to learn the theories but also to practice what you learn by playing lots of games, preferably with players who are stronger than yourself. Before the advent of online mahjong platforms, however, doing so was not very easy if you live outside of Japan. 

\bigskip
Thanks to the recent development of online mahjong platforms, it is now feasible for you to play hundreds or thousands of games with serious opponents while living outside of Japan. On these websites, you can easily find fellow players to play with 24/7. Most platforms keep the record of all the games players have played and a replay function that allows you to reflect on your past plays. You can also take a look at player statistics, which gives you important clues as to what skills you need to work on.  

\bigskip
I thus recommend you practice mahjong skills by playing online while you study the strategy principles with this book. You do not need to wait until you finish reading everything covered in the book before you start playing. Go ahead and play games first, then come back to the book and study the relevant parts of the book. 

\bigskip

This book is divided into four parts. Part \ref{part:online} provides an introduction to an online mahjong platform called {\jap Tenhou} (天鳳). The website is in Japanese, but I will walk you through the account registration process and show you how to play games in Chapter \ref{ch:Tenhou}. There already exist several excellent online resources that explain how to play {\jap Tenhou}, including:
\bi \itemsep-.5em
\i Arcturus's Tenhou Documentation
\vspace{-10pt} \index{Arcturus}
	\bi
	\i[] \url{http://arcturus.su/tenhou/}
	\ei
	
\i Complete Beginner's Guide to Online Mahjong (Osamuko)
\vspace{-10pt} \index{Osamuko}
	\bi
	\i[] \url{https://bit.ly/2CXAKoM}
	\ei
\i Playing Online: Tenhou (Reach Mahjong of New York)
\vspace{-10pt}
	\bi \i[] \url{https://bit.ly/2sd0tU4} \ei
\ei
If you have already read either of the three before, you can skip Chapter \ref{ch:Tenhou} of this book, for there is not much new information there for you. Chapter \ref{ch:Tenhou2} explains some advanced features of {\jap Tenhou}, which you can also skip when you read this book for the first time.

\bigskip
Parts \ref{part:tile} and \ref{part:strat} are the ``meat'' of the book. 
Part \ref{part:tile} covers basic tile efficiency theories that allow you to maximize the speed and/or hand value of your hand. After introducing basic terminology in Chapter \ref{ch:basic}, I discuss the five-block method in Chapter \ref{ch:five} and provide some tips on how to pursue several {\jap yaku} in Chapter \ref{ch:yaku}. 
Part \ref{part:strat} covers strategy principles, including score calculation methods (Chapter \ref{ch:scores}), riichi judgement (Chapter \ref{ch:riichi}), defense judgement (Chapter \ref{ch:defense}), melding judgement (Chapter \ref{ch:call}), and so called ``grand strategies'' to win a game (Chapter \ref{ch:grand}). 
Finally, Appendices include a chapter on etiquettes for offline playing (Chapter \ref{ch:manners}) and another chapter on further readings (Chapter \ref{ch:read}). 

\bigskip
Numbers and letters shown {\color{MyBlue} in this color} as well as each entry in the Contents section below are hyperlinked; clicking on one will take you to the pertinent page. 
%I omit page numbers to save space, but each page is given an implicit page number that a PDF reader (such as the Kindle app) would recognize. Each entry in the Index section at the end of the book also refers to such implicit page numbers. 

